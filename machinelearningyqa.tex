\documentclass{article}
\usepackage{xeCJK}         % 中文支持
\usepackage{graphicx}      % 插图支持
\usepackage{fontspec}
\usepackage{listings}
\usepackage{xcolor} 
\setlength{\parindent}{0pt}     % 字体设置(用于 XeLaTeX)
\setCJKmainfont{KaiTi}    % 设置中文字体,例如宋体(可改为微软雅黑、仿宋等)
\lstset{
  basicstyle=\ttfamily\small,
  breaklines=true,            % 自动换行!
  breakatwhitespace=true,
  frame=single,
  language=Python,
  showstringspaces=false,
  keywordstyle=\color{blue},
  commentstyle=\color{gray},
  stringstyle=\color{orange}
}
\title{机器学习编程课}
\author{蔚全爱}
\date{July 2025}

\begin{document}

\maketitle

\section{经典机器学习算法}

\subsection{python基础知识}

变量赋值及打印

\begin{lstlisting}
    a = 3
    b = "abc"
    print(a)
    a = 4
    print(a)
\end{lstlisting}
\textbf{Remark:} 在 Python 中,等号 `=` 表示赋值而非数学意义的相等;字符串必须使用英文单引号或双引号括起来,否则会被当作变量名。

\noindent\textbf{Output:}
\begin{verbatim}
3
4
\end{verbatim}
\par
列表操作
\begin{lstlisting}
c = ["ml", a, b, ["list"]]
print(c)
print(c[0], c[3][0])
print(c + ["aaa"])
print(c * 2)
c.append("new_element")
c.remove(["list"])
print(c)
\end{lstlisting}
\textbf{Remark:} 列表可以包含不同类型的元素,包括其他列表。通过 `+` 连接或 `*` 重复使用 `append` 方法添加元素,使用 `remove` 方法删除元素。

\textbf{Output:}
\begin{verbatim}
['ml', 4, 'abc', ['list']]
ml list
['ml', 4, 'abc', ['list'], 'aaa']
['ml', 4, 'abc', ['list'], 'ml', 4, 'abc', ['list']]
['ml', 4, 'abc', ['list'], 'new_element']
['ml', 4, 'abc', 'new_element']
\end{verbatim}

元组不可变性
\begin{lstlisting}
Tuple = (1, 2, 3)
print(Tuple)
try:
    Tuple[0] = 5
except TypeError:
    print("元组无法被更改")
\end{lstlisting}
\textbf{Remark:} 元组一旦创建后,其内容不可更改。

\textbf{Output:}
\begin{verbatim}
(1, 2, 3)
元组无法被更改
\end{verbatim}

字典的键值操作
\begin{lstlisting}
Dict = {"name1": "ml", "name2": 1, "name3": [1, 2]}
print(Dict["name2"])
Dict["name4"] = "new_ele"
del Dict["name1"]
print(Dict)
\end{lstlisting}
\textbf{Remark:} 字典是键值对的集合,可以通过键访问对应的值,可动态添加或删除键值对。

\textbf{Output:}
\begin{verbatim}
1
{'name2': 1, 'name3': [1, 2], 'name4': 'new_ele'}
\end{verbatim}

for循环
\begin{lstlisting}
for i in range(1, 10, 1):
    List = []
    for j in range(1, 10, 1):
        if i >= j:
            List.append(f"{i}*{j}={i * j}")
    print(List)
\end{lstlisting}
\textbf{Output:}
\begin{verbatim}
['1*1=1']
['2*1=2', '2*2=4']
['3*1=3', '3*2=6', '3*3=9']
['4*1=4', '4*2=8', '4*3=12', '4*4=16']
['5*1=5', '5*2=10', '5*3=15', '5*4=20', '5*5=25']
['6*1=6', '6*2=12', '6*3=18', '6*4=24', '6*5=30', '6*6=36']
['7*1=7', '7*2=14', '7*3=21', '7*4=28', '7*5=35', '7*6=42', '7*7=49']
['8*1=8', '8*2=16', '8*3=24', '8*4=32', '8*5=40', '8*6=48', '8*7=56', '8*8=64']
['9*1=9', '9*2=18', '9*3=27', '9*4=36', '9*5=45', '9*6=54', '9*7=63', '9*8=72', '9*9=81']
\end{verbatim}

简单的for循环
\begin{lstlisting}
eexample = range(10)
example1 = [i + 1 for i in example]
example2 = [i * 2 for i in example]
print(example1)
\end{lstlisting}

\textbf{Output:}
\begin{verbatim}
[1, 2, 3, 4, 5, 6, 7, 8, 9, 10]
\end{verbatim}

字典内容遍历
\begin{lstlisting}
# 遍历字典内容
for i in Dict:
    print(i)
for i in Dict.keys():
    print(i)
for i in Dict.values():
    print(i)
for i in Dict.items():
    print(i)
for i in Dict:
    print(i, Dict[i])
for key, value in Dict.items():
    print(key, value)
\end{lstlisting}

\textbf{Remark:} 遍历字典时默认取键,可通过 \texttt{.keys()}, \texttt{.values()}, \texttt{.items()} 访问不同元素结构。

\textbf{Output:}
\begin{verbatim}
name2
name3
name4
name2
name3
name4
1
[1, 2]
new_ele
('name2', 1)
('name3', [1, 2])
('name4', 'new_ele')
name2 1
name3 [1, 2]
name4 new_ele
\end{verbatim}

if-else语句:

\begin{lstlisting}
if isinstance(a, int):
    print("a 是 int 类型")
elif isinstance(a, str):
    print("a 是 str 类型")
else:
    print("a 不是 int 类型或 str 类型")

if 5 >= a >= 0:
    print(True)
else:
    print(False)
\end{lstlisting}

\textbf{Remark:} 使用 \texttt{isinstance()} 可判断变量类型;Python 支持链式比较,如 \texttt{5 >= a >= 0}。

\textbf{Output:}
\begin{verbatim}
a 是 int 类型
True
\end{verbatim}

方法及调用

\begin{lstlisting}
def method1(a):
    if isinstance(a, int):
        print("a 是 int 类型")
    elif isinstance(a, str):
        print("a 是 str 类型")
    else:
        print("a 不是 int 类型或 str 类型")

method1(a=1)
\end{lstlisting}

类实例化

\begin{lstlisting}
class person:
    leg = 4
    eye = 2

a = person()
a.brain = "smart"

class MyClass:
    i = 12345
    def f(self, b):
        print(self.i)
        return 'hello world'

# 实例化类
x = MyClass()
x.i = 123
# 访问类的属性和方法
print("MyClass 类的属性 i 为:", x.i)
print("MyClass 类的方法 f 输出为:", x.f(b=2))
\end{lstlisting}

\textbf{Remark:}  
类属性可直接访问;实例可动态添加属性(如 \texttt{brain});最后调用f(b=2)时只需要传入参数b,不需要传入self,因为Python会自动将实例作为第一个参数传递给方法。

\begin{lstlisting}
class Circle:
    pi = 3.1415
    #特殊的函数,调用的时候会自动调用
    def __init__(self, r):
        self.r = r #注册的类的属性,只有实例化后才能访问

    def print_r(self):
        print(self.r)

    def cal(self, Type):
        pass

circle1 = Circle(r=1) #实例化为circle1
circle1.r = 2
circle1.pi = 3.14
print(Circle.pi, circle1.pi)
\end{lstlisting}

\textbf{Remark:}  
\texttt{\_\_init\_\_} 构造函数会在实例化时自动调用;类属性如 \texttt{pi} 可通过实例或类名访问,但可被实例覆盖; circle1.pi = 3.14 修改了实例的 \texttt{pi} 属性,但不会影响类属性。

\textbf{Output:}
\begin{verbatim}
3.1415 3.14
\end{verbatim}

类的继承

\begin{lstlisting}
class Ellipse(Circle):#继承自哪个类,在原类上增加不同的特征
    def __init__(self, r, r1):
        super(Ellipse, self).__init__(r)
        self.r1 = r1

    def cal(self, Type):
        if Type == "Circumference":
            return 2 * self.pi * self.r1 + 4 * (self.r - self.r1)
        elif Type == "Area":
            return 2 * self.pi * self.r1 * self.r
        else:
            raise NotImplementedError

ell = Ellipse(2, 3)
result = ell.cal("Area")
print(result)
ell.print_r()
\end{lstlisting}

\textbf{Remark:}  
\texttt{Ellipse} 类继承 \texttt{Circle};\texttt{super()} 可调用父类构造函数和父类的self.f一致;子类可添加新属性(如 \texttt{r1})

\textbf{Output:}
\begin{verbatim}
37.698
2
\end{verbatim}

\subsection{基础算法}

Kmeans聚类算法的sklearn实现

\begin{lstlisting}
import numpy as np
import sklearn.datasets as ds
from sklearn.cluster import KMeans, DBSCAN
from sklearn.metrics import homogeneity_score, completeness_score, v_measure_score, \
    adjusted_mutual_info_score, adjusted_rand_score, silhouette_score
import matplotlib as mpl
import matplotlib.pyplot as plt

# 生成数据
x, y = ds.make_blobs(n_samples=400, n_features=2, centers=4, random_state=2025)

# KMeans 聚类
model = KMeans(n_clusters=4)
model.fit(x)
y_pred = model.predict(x)

# 输出聚类评估指标
print('y_true = ', y)
print('y_pred = ', y_pred)
print('homogeneity_score = ', homogeneity_score(y, y_pred))
print('completeness_score = ', completeness_score(y, y_pred))
print('v_measure_score = ', v_measure_score(y, y_pred))
print('adjusted_mutual_info_score = ', adjusted_mutual_info_score(y, y_pred))
print('adjusted_rand_score = ', adjusted_rand_score(y, y_pred))
print('silhouette_score = ', silhouette_score(x, y_pred))

# 可视化结果
plt.figure(figsize=(8, 4))

plt.subplot(121)
plt.plot(x[:, 0], x[:, 1], 'r.', ms=3)

plt.subplot(122)
plt.scatter(x[:, 0], x[:, 1], c=y_pred, marker='.', cmap=mpl.colors.ListedColormap(list('rgbm')))

plt.tight_layout()
plt.show()
\end{lstlisting}
\textbf{Output visualization:}

\begin{figure}[h]
\centering
\includegraphics[width=0.8\textwidth]{output.png}
\caption{KMeans 聚类可视化结果:左图为原始数据,右图为聚类结果}
\end{figure}

Kmeans聚类算法的手动实现:随机初始化聚类中心,分配每个样本到最近的聚类中心,更新聚类中心,直到收敛。

\begin{lstlisting}
import numpy as np

class KMeans:
    def __init__(self, n_clusters=3, max_iter=100):
        self.n_clusters = n_clusters
        self.max_iter = max_iter

    def fit(self, X):
        # 随机初始化聚类中心
        self.centroids = X[np.random.choice(X.shape[0], self.n_clusters, replace=False)]
        for _ in range(self.max_iter):
            # 分配每个样本到最近的聚类中心
            distances = np.linalg.norm(X[:, np.newaxis] - self.centroids, axis=2)
            labels = np.argmin(distances, axis=1)
            # 更新聚类中心
            new_centroids = np.array([X[labels == i].mean(axis=0) for i in range(self.n_clusters)])
            if np.all(new_centroids == self.centroids):
                break
            self.centroids = new_centroids
        return self

    def predict(self, X):
        distances = np.linalg.norm(X[:, np.newaxis] - self.centroids, axis=2)
        return np.argmin(distances, axis=1)
\end{lstlisting}

完整的KMeans聚类算法实现

\begin{lstlisting}
import numpy as np
import pandas as pd
from sklearn.preprocessing import StandardScaler, PolynomialFeatures
from sklearn.linear_model import LogisticRegression
from sklearn.pipeline import Pipeline
import matplotlib.pyplot as plt

if __name__ == "__main__":
    path = './iris.data'  # 数据文件路径
    data = pd.read_csv(path, header=None)
    data[4] = pd.Categorical(data[4]).codes  # 类别标签转为整数
    x, y = np.split(data.values, (4,), axis=1)  # 前4列是特征,最后1列是标签

    # 仅使用前两列特征
    x = x[:, :2]

    # 构建 Pipeline:标准化 + 多项式特征 + 逻辑回归,先后顺序执行
    lr = Pipeline([
        ('sc', StandardScaler()),
        ('poly', PolynomialFeatures(degree=10)),
        ('clf', LogisticRegression())
    ])

    lr.fit(x, y.ravel())  # 训练模型
    y_hat = lr.predict(x)  # 预测类别
    y_hat_prob = lr.predict_proba(x)  # 预测概率

    np.set_printoptions(suppress=True)
    print('y_hat = \n', y_hat)
    print('y_hat_prob = \n', y_hat_prob)
    print("准确率: %.2f%%" % (100 * np.mean(y_hat == y.ravel())))

    # 可视化决策边界
    N, M = 200, 200  # 横纵各采样多少个点
x1_min, x1_max = x[:, 0].min(), x[:, 0].max()  # 第0列的范围
x2_min, x2_max = x[:, 1].min(), x[:, 1].max()  # 第1列的范围

t1 = np.linspace(x1_min, x1_max, N)
t2 = np.linspace(x2_min, x2_max, M)
x1, x2 = np.meshgrid(t1, t2)  # 生成网格采样点
x_test = np.stack((x1.flat, x2.flat), axis=1)  # 测试点坐标集

# 设置中文字体、颜色
mpl.rcParams['font.sans-serif'] = ['simHei']
mpl.rcParams['axes.unicode_minus'] = False
cm_light = mpl.colors.ListedColormap(['#77E0A8', '#FF8080', '#A0A0FF'])  # 背景色
cm_dark = mpl.colors.ListedColormap(['g', 'r', 'b'])                     # 点颜色

# 预测新样本的类别
y_hat = lr.predict(x_test)
y_hat = y_hat.reshape(x1.shape)  # 转为网格形状

# 绘图
plt.figure(facecolor='w')
plt.pcolormesh(x1, x2, y_hat, cmap=cm_light)  # 预测区域图
plt.scatter(x[:, 0], x[:, 1], c=y.ravel(), edgecolors='k', s=50, cmap=cm_dark)  # 样本点图
plt.xlabel('花萼长度', fontsize=14)
plt.ylabel('花萼宽度', fontsize=14)
plt.title('逻辑回归对Iris数据集的分类结果', fontsize=16)
plt.tight_layout()
plt.show()
\end{lstlisting}

\begin{figure}[htbp]
    \centering
    \includegraphics[width=0.8\textwidth]{output2.png}
    \caption{逻辑回归在 Iris 数据集上的分类结果与决策边界}
    \label{fig:logistic-iris}
\end{figure}
\end{document}
